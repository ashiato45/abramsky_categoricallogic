\documentclass[9pt]{ltjsarticle}
\DeclareSymbolFont{bbold}{U}{bbold}{m}{n}
\DeclareSymbolFontAlphabet{\mathbbold}{bbold}
\newcommand{\bbold}{\mathbbold}
\usepackage{xcolor}
\usepackage{amsmath,amsfonts,amssymb}
\usepackage{enumitem}
\usepackage{ashiato45}
%\usepackage{okumacro}
\def\MARU#1{\textcircled{\scriptsize #1}}
\usepackage{graphicx}
\usepackage{ulem}
\usepackage{framed}
\usepackage{algorithm}
\usepackage{algorithmic}
\usepackage{here}
%\usepackage[twoside]{geometry}
\usepackage{mytheorems}
\usepackage{tikz}
\usepackage{ascmac}
\usetikzlibrary{cd}
\title{Abramsky- Categorical Logic}
\author{ashiato45 take notes}

\renewcommand{\bf}{\mathbf}


\begin{document}
\maketitle

\section{Introduction}
\label{sec:Introduction}
\subsection{From Elements To Arrows}
\label{sub:From Elements To Arrows}
\subsection{Categories Defined}
\label{sub:Categories Defined}
\subsection{Diagrams in Categories}
\label{sub:Diagrams in Categories}
\subsection{First Notions}
\label{sub:First Notions}
\begin{itemize}
  \item (Definition 15: subcategory):
  $\C$を圏とし、
  \begin{align}
    \obj(\D)\subset \obj(\C),\quad
    \Forall{A,B\in \obj(\cal D)} \cal D(A,B) \subset \cal C(A,B)
  \end{align}
  がるとする。$\D$が$\C$のsubcategoryであるとは、
  \begin{itemize}
    \item $A\in \obj(\D) \implies \id_A \in \cal D(A,A)$となる
    \item $f\in \cal D(A,B),g\in \cal D(B,C)\implies g\circ f \in \cal D(A,C)$
  \end{itemize}
  となること。
  \item (full,lluf):
  subcategoryがfullであるとは、$A$から$B$への射が
  もとのcategoryのものと一致すること。
  llufであるとは、オブジェクトをすべて引き継いでいること。

\end{itemize}

\section{Some Basic Constructions}
\label{sec:Some Basic Constructions}
\subsection{Initial and Terminal Objects}
\label{sub:Initial and Terminal Objects}
\subsection{Products and Coproducts}
\label{sub:Products and Coproducts}
\subsubsection{Products}
\label{subs:Products}
\subsection{Coproducts}
\label{sub:Coproducts}
\subsection{Pullbacks and Equalisers}
\label{sub:Pullbacks and Equalisers}
\subsubsection{Pullbacks}
\label{subs:Pullbacks}
\begin{itemize}
  \item
  $A\xrightarrow{f} C \xleftarrow{g}B$について、
  $f,g$に沿ったpullbackとは、$A\xleftarrow{p}D \xrightarrow{q} B$で、
  \begin{quote}
    \begin{tikzcd}
      D' \ar[rd,dotted,"h"] \ar[rrd,"q'"] \ar[rdd,"p'"]& & \\
      & D \ar[r,"q"] \ar[d,"p"] & B \ar[d,"g"]\\
      & A\ar[r,"f"] & C
    \end{tikzcd}
  \end{quote}
  という普遍性を持つもの。
  \item
  ($(f,g)$-cone):
  $A\xrightarrow{f}C\xleftarrow{g} B$について、$(f,g)$-oneとは
  $(D,p,q)$で
  \begin{tikzcd}
    D\ar[r,"q"] \ar[d,"p"] & B \ar[d,"g"]\\
    A \ar[r,"f"] & C
  \end{tikzcd}
  となるもの。
  \item
  ($(f,g)$-coneのmorphism):
  $A \xrightarrow{f} C \xleftarrow{g}B $についてこのconeたちを考える。
  $(f,g)$-coneである、$(D_1,p_1,q_1),(D_2,p_2,q_2)$について、
  この間のmorphismとは、
  \begin{tikzcd}
    & D_1 \ar[ld,"p_1"] \ar[dd,"h"] \ar[dr,"q_1"]& \\
    A & & B\\
    & D_2 \ar[ul,"p_2"] \ar[ur,"q_2"]&
  \end{tikzcd}
  となるもの。
\end{itemize}
\subsubsection{Equalisers}
\label{subs:Equalisers}
\begin{itemize}
  \item (Definition 28: equaliser):
  \begin{tikzcd}
    A  \ar[r,bend left,"f"] \ar[r,bend right,"g"]& B
  \end{tikzcd}
  を考える。
  $(f,g)$のequaliserとは、
  \begin{itemize}
    \item arrow$e\colon E\to A$で、
    \item $f\circ e = g\circ e$をみたし、
    \item
    \begin{tikzcd}
      E\ar[r,"e"] & A \ar[r,bend left,"f"] \ar[r,bend right, "g"] & B\\
      D \ar[u,dotted,"\hat h"] \ar[ru,"h"] & &
    \end{tikzcd}
    という普遍性を持つもの。
  \end{itemize}
  \item
  たとえば、$f\colon (x,y)\mapsto x^2 + y^2$、$g\colon (x,y)\mapsto 1$
  について、単位円からの$\R^2$への埋め込みはequaliserになっている。
\end{itemize}
\subsection{Limits and Colimits}
\label{sub:Limits and Colimits}

\section{Functors}
\label{sec:Functors}
\subsection{Basics}
\label{sub:Basics}
\subsection{Further Examples}
\label{sub:Further Examples}
\begin{itemize}
  \item posetの例:
  poset\footnote{反射対称推移}$P$が時間をあらわすとき、
  $F\colon P \to \bf{Set}$は集合の時間変化をあらわす。
  \item
  $\bf{List}$はfunctorになる。
  つまり、$f\colon X \to Y$を
  $\bf{List}(X)\to \bf{List}(Y)$にうつす:
  \horrel{$f\colon X\to Y$}{$\bf{List}(f)\colon \bf{list}(X)\to \bf{list}(Y)$}。
  \item
  さらに構造を持たせることもできる。
  Listをmonoidとみなせば、
  $\bf{MList}\colon \bf{Set}\to \bf{Mon}$を上と
  同様に定義すればよい。Listは単位元を空リストとし、
  積を連結とすればmonoidになっている。
  \item (free monoid):
  $\bf{MList}(X)$を$X$上のfree monoidという。
\end{itemize}


\subsection{Contravariance}
\label{sub:Contravariance}

\subsection{Properties of Functors}
\label{sub:Properties of Functors}
\begin{itemize}
  \item (Definition 48: faithful):
  functorの、arrowをうつす部分が単射。
  \item (full):
  functorの、arrowをうつす部分が全射。
  \item (embedding):
  functorがfullでfaithfulで「objectに関して単射」である。
  \item (essentially surjective):
  \begin{align}
    \Forall{B\in \obj(\cal D)} \Exists{A\in \obj(\cal C)} F(A) \simeq B
  \end{align}
  つまり、「『objectについての全射』を同型で弱めた」もの。
  \item (equivalence):
  functorがfullでfaithfulでessentailly surjectiveっであること。
  \item (isomorphism):
  合成して1。
  \item (preservation)
  $P$をarrowの性質とする
  \footnote{monicとかepiとか}。
  $F\colon \cal C\to \cal D$が$P$をpreserveするとは、
  $f$が$P$をみたすとき$F(f)$も$P$をみたすこと。
  \item (reflect):
  $F(f)$が$P$をみたすとき$f$も$P$をみたすこと。
\end{itemize}

\section{Natural Tansformations}
\label{sec:Natural Tansformations}
\subsection{Basics}
\label{sub:Basics}

\begin{itemize}
  \item (natural isomorphism):
  自然変換$t\colon F\to G$について、
  \begin{align}
    \Forall{A\in \obj(\cal C)} t_A は\text{isomorphism}
  \end{align}
  となるとき、$t$をnatural isomorphismという。
  \item
  Functor$\id$と、Functor$\times \circ \gen{\id,\id}$
  (これは$f\mapsto \gen{f,f}$と定義される)の間の
  $\Delta \colon \id \to \times \circ \gen{\id,\id}$
  を$X\in \bf{Set}$について、
  \begin{align}
    \Delta_X \colon x\to X\times X ,\quad x\mapsto (x,x)
  \end{align}
  と定義し、これを
  考える。これは、
  \begin{tikzcd}
    X\ar[r,"f"] & Y\\
    X\ar[r,"f"] \ar[d,"\Delta_X"] & Y \ar[d,"\Delta_Y"]\\
    X\times X \ar[r,"f\times f"]& Y \times Y
  \end{tikzcd}
  をcommuteさせるので、自然変換になっている。
  \item
  binary productsがある圏$\C$について、
  上と同様にfunctor$\id$からfunctor$\id \times \id$
  への自然変換$\Delta_A$が定義できる。
  \item
  binary productsがある圏$\C$について、
  functor $\times\colon \C\times \C \to \C$をproduct categoryからとってその2つのbinary productsをとるものとし、
  functor $\pi_1\colon \C\times \C \to \C$をproduct categoryをとってその片方をとるものとする。
  このとき、この2つのfunctorの間のtransformation $\pi_1$
  を、各$(A,B)\in \obj(\cal C\times \cal C)$について、
  \begin{align}
    (\pi_1)_{(A,B)}\colon
    (A,B) \mapsto A
  \end{align}
  と定義すると、これはnatural transformationになる。
  \item
  $\C$をterminal $T$をもつ圏とし、
  functor $K_T\colon \cal C\to \cal C$を
  、すべてのobjectを$T$にうつし、すべての射を$\id_T$にうつすものとする。
  すると、$\id$から$K_T$へのtransformation を
  \begin{align}
    \tau_A \colon c\mapsto (\id c\to K_T c) = (c\to T)
  \end{align}
  を、$T$がterminalであることから$c\to T$が1つに定まる
  ことを使って定義する。すると、
  あきらかに
  \begin{tikzcd}
    c \ar[d,"f"] & \id c = c \ar[d,"\id f = f"] \ar[r,"\tau c"] & K_T c = T \ar[d,"K f = \id_T"]\\
    c' & \id c' = c' \ar[r,"\tau c'"] & K_T c' =T
  \end{tikzcd}
  の図示が満たされるので(重ね重ねterminalの性質に注意)、
  $\tau$はnatural transformationになっている。
  \item
  $\bf{List}$間のtransformation
  $\bf{List}X \xrightarrow{\bf{reverse}_X} \bf{List}X$
  をリストの逆転と定義すると、これは
  \begin{tikzcd}
    X \ar[d,"f"] & \bf{List}X \ar[d,"\bf{List}f"] \ar[r,"\bf{reverse}_X"] & \bf{List}X \ar[d,"\bf{List}f"]\\
    X' & \bf{List}X' \ar[r,"\bf{reverse}_{X'}"] & \bf{List}X'
  \end{tikzcd}
  をあきらかに満たし、natural。
  \item
  $\id$から$\bf{List}$へのtransformation
  $\id X \xrightarrow{\bf{unit}_X} \bf{List} X$を
  1要素リストの生成と定義すると、これはあきらかにnatural。
  \item
  $\bf{List}(\bf{List}(X))$から$\bf{List}(X)$への
  transformationをリストをつぶすことと定義すると、
  $\bf{List}$のネストで矢がどこへ飛ぶかを考えるとnatural。
  \item
  functor $P\colon \bf{Mon}\to \bf{Set}$を
  、monoidを忘れつつ対角線に埋め込む
  $(M,\cdot,1)\mapsto M\times M,\, f\mapsto f\times f$
  と定義する。さらに、$U$を単にforgetするfunctorとする。
  このとき、$P$から$U$へのtransformation$t_{(M,\bullet,1)}$を
  もとのモノイドでの積を取るとすると、
  \begin{tikzcd}
    (M,\bullet,1) \ar[d,"f"] & M\times M \ar[d,"f\times f"] \ar[r,"t_{(M,\bullet,1)}"] & M \ar[d,"f"]\\
    (N,\bullet,1) & N\times N  \ar[r,"t_{(N,\bullet,1)}"] & N
  \end{tikzcd}
  がcommuteし、natural。

\end{itemize}

\subsection{Further Examples}
\label{sub:Further Examples}
\begin{itemize}
  \item
  $\C$の射$f\colon A\to B$を考える。
  このとき、$\hom_\C (B,?)$から$\hom_\C (A,?)$への
  transformation $\hom_{\C}(f,?)\colon \hom_\C(B,C) \to \hom_\C(A,C)$
  を、「コドメイン側に$f$を合成する」とすると、
  \begin{tikzcd}
    C \ar[d,"h"] & \hom_\C(B,C) \ar[d,"h\circ ?"] \ar[r,"?\circ f"] & \hom_\C(A,C) \ar[d,"h\circ ?"]\\
    D & \hom_\C(B,D) \ar[r,"?\circ f"] & \hom_\C(A,D)
  \end{tikzcd}
  はあきらかにcommuteでnaturalになる。
  \item (Lemma 1;Yoneda Lemma):
  (どのHom-functor間のnatural transformationは矢の合成としてあらわせる)
  $A,B\in \obj(\C)$とする。$t$を$\hom(A,?)$から$\hom(B,?)$への
  natural transformationとする。
  $t= (\circ f)$となる$f\colon B\to A$がただ1つ存在する。

  \pf
  $f=t_A(\id_A)$とすればよいことを示す。

  $C\in \obj(\C)$をfixする。$t(C)=(\circ f)(C)$を示せばよい
  (この型は$\hom(A,C)\to \hom(B,C)$になっている)。
  関手$\hom(A,?),\hom(B,?)$は圏$\C$から$\bf{Set}$への
  関手なので、射は関数になっている。したがって、
  外延的に等価性を示せる。
  $g\colon A\to C$を$\ar(\C)$からfixする。
  $t(C)(g)=(\circ f)(C)(g)$を示せばよい。
  $g$に$t$のnaturalityを使うと、
  \begin{tikzcd}
    A \ar[d,"g"] & \hom(A,A) \ar[d,"g\circ"]\ar[r,"t_A"] & \hom(B,A) \ar[d,"g\circ"]\\
    C & \hom(A,C) \ar[r,"t_C"] & \hom(B,C)
  \end{tikzcd}
  を得る。
  \begin{align}
    t_C(g) = t_C((g\circ ?)(\id_A))
    \desceq{comm}
    (g\circ ?)t_A(\id_A)
    \desceq{$f$の定義}
    (g\circ ?)f
    =
    g\circ f
    =
    (\circ f)g
    .
  \end{align}
  $g$のfixを外し、$t_C = (\circ f)=(\circ f)(C)$となる。
  よって、$t = (\circ f)$となる。示された。
  \item
  (Definition 57; equivalent)
  圏$\C,\D$がequivalentっであるとは、
  関手$F\colon \C\to \D,G\colon \D \to \C$があり、
  $G\circ F \simeq \id_{\C},\, F\circ G \simeq \id_{\D}$となる。
\end{itemize}

\subsection{Functor Categories}
\label{sub:Functor Categories}

\section{Universality and Adjoints}
\label{sec:Universality and Adjoints}
\subsection{Adjunctions for Posets}
\label{sub:Adjunctions for Posets}
\begin{itemize}
  \item ($g$-approximation of $x$):
  $x\in P$とする。
  $P,Q$をPosetの圏のobject(つまりposet)とする。
  $g\colon Q\to P$をposet準同型とする。
  $x\le g(y)$となる$y\in Q$を$x$の
  $g$-approximationという。
  \item (best $g$-approximation of $x$):
  $y\in Q$で
  \begin{align}
    x\le g(y) \land (x\le g(z) \implies y\le z)
  \end{align}
  \item
  $g$が全射であり、したがって
  常に$g$-approximationがあるとしても
  bestなものがあるとは限らない。これが
  canonical choiceの問題になる。
  \item (posetのleft adjoint):
  もしも全ての$x\in P$について
  そのbest $g$-approximationがあるなら、
  それでもって$f\colon P\to Q$を定義できる。
  この$f$を$g$のleft adjointという。
  このとき、
  \begin{align}
    x\le g(z)\iff f(x)\le z
  \end{align}
  となっている。
  \item (binary relationのleft adjoint):
  二項関係の圏をかんがえる。$R\subset X\times Y$とする。
  \begin{align}
    f_R\colon \cal P(X)\to \cal P(Y),\quad
    S\mapsto \bigcup_{x\in S, xRy}y
  \end{align}
  $f_R$はright adjoint $[R]\colon \cal P(Y)\to \cal P(X)$
  は$S,T\subset \cal P(Y)$としておいて、
  \begin{align}
    S\subset [R]T \iff f_R(S)\subset T
  \end{align}
  をみたしてほしいが、これは
  \begin{align}
    [R]T
    \defeq
    \set{x\in X; xRy \implies y\in T}
  \end{align}
  である。
  \item (powersetのleft adjoint):
  $f\inv \colon \cal P(Y)\to \cal (X)$を考えることができるが、
  left adjoint$\exists(f)\colon \cal P(X)\to \cal P(Y)$
  とright adjoint$\forall(f)\colon \cal P(X)\to \cal P(Y)$
  、すなわち$S\subset X,T\subset Y$について、
  \begin{align}
    \exists(f)(S) \subset T
    \iff
    S\subset f\inv (T),\quad
    f\inv(T)\subset S
    \iff
    T\subset \forall(f)(S).
  \end{align}
  これは、$\exists(f)$を$S$の像とし、
  $\forall(f)(S)$を「$y$に行く$x$は全部$S$上」となるような$y$たちとする。
  つまり、$x\in S$から$f$で飛ばした先の点$f(x)$が、$S$の外から来てはいけない。
\end{itemize}

\subsection{Universal Arrows and Adjoints}
\label{sub:Universal Arrows and Adjoints}
\begin{itemize}
  \item (Definition 64:universal arrow)
  $G\colon \D \to \C$をfunctorとする。
  $C\in \obj(\C)$から$G$へのuniversal arrowとは、
  $(D \in \obj(\D),\eta \in \hom_\C(C,G(D))$の組で、
  \begin{quote}
    $\Forall{D' \in \obj(\D)}\Forall{f\colon C\to G(D')}$
    \begin{tikzcd}
      C \ar[r,"\eta"] \ar[rd,"f"] & G(D) \ar[d,dotted,"G(\hat f)"] & D \ar[d,"\hat f",dotted] \\
      & G(D') & D'
    \end{tikzcd}
  \end{quote}
  をみたすもの。

  一意性の条件は式で
  \begin{align}
    \Forall{h\colon D\to D'} \hat{G(D\circ \eta)} = h
  \end{align}
  ともあらわせる。
  \pf
  $h\colon D\to D'$で、$f=G(D)\circ \eta$なら$\hat f=h$であることを示せばよいが、
  ハットのなかを置き換えて、これは$\hat{G(D)\circ \eta}=h$である。

  \item
  $U$をmonoidから集合へのforgetする関手とする。
  このとき、$X$から$U$へのuniversal arrowとして、
  $(\bf{MList}(X),\eta_X)$がとれる。ただし、$\eta_X\colon X\to U(\bf{MList}(X))$
  は1要素リストの生成。
  \pf
  任意の集合圏での関数$f\colon X\to M$について
  \begin{align}
    \hat f\colon
    \bf{MList}(X) \to (M,\bullet, 1)
    ,\quad
    [x_1,\dots,x_n] \mapsto f(x_1)\dots f(x_n)
  \end{align}
  が存在して、これが
  \begin{tikzcd}
    {\bf{Set}の話} & & {\bf{Mon}の話}\\
    X\ar[r,"\eta_X"] \ar[rd,"f"] & {U(\bf{MList}(X))} \ar[d,dotted,"U(\hat f)"] &  \bf{MList}(X) \ar[d,dotted,"\hat f"]\\
    & M & (M,\bullet,1)
  \end{tikzcd}
  の条件をみたす。実際、
  \begin{align}
    (U(\hat f)\circ \eta_X)(x) = U(\hat f)([x]) = f(x)
  \end{align}
  である。一意性は、上の条件より
  $\hat{U(h)\circ \eta_X}=h$を示せばよい。
  \begin{align}
    \hat{U(h)\circ \eta_X}([x_1,\dots,x_n])
    &=
    (U(h)\circ \eta_X)(x_1)\cdot \dots \cdot (U(h)\circ \eta_X)(x_n)\\
    &=
    U(h)([x_1])\cdot \dots \cdot U(h)([x_n])\\
    &=
    h([x_1])\dots h([x_n])\\
    &=
    h([x_1,\dots,x_n]).
  \end{align}
  \item
  (Definition 68:adjunction,adjoint)
  $\C,\D$を圏とする。$\C$から$\D$へのadjunctionは
  \begin{itemize}
    \item $F$:$\C$から$\D$へのfunctor
    \item $G$:$\D$から$\C$へのfunctor
    \item $\theta$: $\C$のobject$A,B$で添字付けられたbijectionのfamily
    \begin{align}
      \theta_{A,B} \colon \C( A, G(B)) \xrightarrow{\simeq} \D(F(A),B)
    \end{align}
    で、$A$をrunしたとき自然になり、$B$をrunしたとき自然になるもの。
    \item
    $F$を$G$のleft adjointといい、$G$を$F$のright adjointという。
  \end{itemize}
  \item
  上の2つの自然性について考えてみる。
  まず、$A$を固定し$B$を走らせたときのnaturalityは
  \begin{tikzcd}
    B \ar[d,"h"] & \C (A,GB) \ar[d,"(Gh)\circ"] \ar[r,"\theta_{AB}"]&  \D(FA,B) \ar[d,"h\circ"]\\
    B' & \C (A,GB') \ar[r,"\theta_{AB'}"] & \D (FA,B')
  \end{tikzcd}
  となる。
  \item
  一方、$B$を固定し$A$を走らせたときのnaturalityは
  \begin{tikzcd}
    A' \ar[d,"g"] & \C (A',GB) \ar[r,"\theta_{A'B}"]  & \D (FA',B)\\
    A & \C (A,GB) \ar[u,"\circ g"] \ar[r,"\theta_{AB}"] & \D (FA,B) \ar[u,"\circ (Fg)"]
  \end{tikzcd}
  となる。
  \item
  上2つをくっつけると、
  \begin{tikzcd}
    A' \ar[d,"g"] & & \C(A',GB) \ar[r,"\theta_{A'B}"] & \D (FA',B) \\
    A & B\ar[d,"h"] & \C (A,GB) \ar[u,"\circ g"] \ar[r,"\theta_{AB}"] \ar[d,"(Gh)\circ"] & \D (FA,B) \ar[u,"\circ (Fg)"] \ar[d,"h\circ"]\\
    & B' & \C (A,GB') \ar[r]&\D (FA,B')
  \end{tikzcd}
  となる。これがnaturalityの意味になる。
  さらに、$B$を固定し$A$を走らせたときの図式で
  $B'$を固定した図式を考えくっつけると、
  \begin{tikzcd}
    B\ar[d,"h"] & & \C (A,GB) \ar[r,"\theta_{AB}"] \ar[d,"(Gh)\circ"] & \D (FA,B)\ar[d,"h\circ"]\\
    B' & A & \C (A,GB') \ar[r,"\theta_{AB'}"] \ar[d,"\circ g"] & \D (FA,B') \ar[d,"\circ (Fg)"]\\
    & A' \ar[u,"g"] & \C (A',GB') \ar[r,"\theta_{A'B'}"] & \D (FA',B')
  \end{tikzcd}
  を得る。

  中間を引っこ抜くと、
  \begin{tikzcd}
    B \ar[d,"h"] & A & \C (A,GB) \ar[r,"\theta_{AB}"] \ar[d,"(Gh)\circ ? \circ g"] & \D (FA,B) \ar[d, "h\circ ? \circ (Fg)"]\\
    B' & A' \ar[u,"g"] & \C (A',GB') \ar[r,"\theta_{A'B'}"] & \D (FA',B')
  \end{tikzcd}
  を得る。
  これは$f,g$を$\id$にしたりすることで上の式をすべて包含する。
\end{itemize}

\end{document}
